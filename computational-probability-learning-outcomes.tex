\documentclass{article}
\usepackage[margin=1in]{geometry}
\newcommand{\lot}[2]{
    \begin{tabular}[t]{p{.60\linewidth}p{.30\linewidth}}\raggedright
        #1 & \begin{minipage}[t]{\linewidth}\raggedright
            \begin{itemize}  \setlength\itemsep{0em}
                #2
            \end{itemize}
        \end{minipage}
    \end{tabular}
}
\newcommand{\lo}[1]{
    \begin{tabular}[t]{p{.60\linewidth}p{.30\linewidth}}\raggedright
        #1 & 
    \end{tabular}
}
\begin{document}\raggedright
\section*{Enduring understanding}
The following are the four ideas that I hope will persist with students after the minutia of the Poisson distribution has faded from memory.  
\begin{enumerate}
    \item Probability is a framework for organizing beliefs; it is not a statement of what your beliefs should be.
    \item Probability is a framework for coherently updating beliefs based on new information and data.
    \item Probability models are a powerful framework for describing and simplifying real world phenomena as a means of answering research questions.
    \item Probability models can be expressed and applied mathematically and computationally.
\end{enumerate}


\section*{Learning outcomes}
The learning outcomes are listed below with course topics listed to the right.
\begin{enumerate}
    \item Probability is a framework for organizing beliefs; it is not a statement of what your beliefs should be.

    \begin{enumerate}
        \item[] \hspace*{\dimexpr-\labelsep-\labelwidth}Students will ...
        \item \lot{compare and contrast different definitions of probability, illustrating differences with simple examples}{\item long-run proportion \item personal beliefs \item combination of beliefs and data}
        \item \lo{verbally express the rules of probability.}
        \item \lo{mathematically express the rules of probability.}
        \item \lo{computationally express the rules of probability.}
        \item \lo{illustrate the rules of probability with examples.}
        \item \lo{using the long-run proportion definition of probability, derive the univariate rules of probability.}
        \item \lo{organize/express bivariate random variables in cross tables.}
        \item \lo{define joint, conditional, and marginal probabilities.}
        \item \lo{identify J, C, and M probabilities in cross tables.}
        \item \lo{identify when a research question calls for a J, C, or M probability.}
        \item \lo{describe the connection between conditional probabilities and prediction.}
        \item \lo{derive Bayes rule from cross tables.}
        \item \lo{apply Bayes rules to answer research questions.}
        \item \lo{determine if joint outcomes are independent.}
        \item \lo{calculate a measure of association between joint outcomes.}
    \end{enumerate}
    \item Probability is a framework for coherently {\bf updating} beliefs based on new information and data.
    \item Probability models are a powerful framework for describing and simplifying real world phenomena as a means of answering research questions.
    \item Probability models can be expressed and applied mathematically and computationally.
\end{enumerate}

\end{document}





2.	


        <details><summary>Topics</summary>

        * Complement (`!%in%`)
        * AND (`&`)
        * OR (`|`)
        * Total probability

    </details>
1. Using the long-run proportion definition of probability, students will derive the univariate rules of probability.
1. Students will be able to organize/express bivariate random variables in cross tables.
1. Students will define joint, conditional, and marginal probabilities.
1. Students will identify J, C, and M probabilities in cross tables.
    1. Students will identify when a research question calls for a J, C, or M probability.
    1. Students will describe the connection between conditional probabilities and prediction.
1. Student will derive Bayes rule from cross tables.
1. Students will be able to apply Bayes rules to answer research questions.
1. Students will be able to determine if joint outcomes are independent.
1. Students will be able to calculate a measure of association between joint outcomes.
1. Students will be able to define and use the special terminology 
    <details><summary>Topics</summary>

    * Positive predictive value
    * Negative predictive value
    * Combination of data and belief

    </details>  
</details>
<details>
<summary>2. </summary>
</details>
<details>
<summary>3. </summary>

1. Student will be able to list various data types
    1. Students will be able to match each data type with probability models that may describe it.
    1. Students will be able to discuss the degree to which models describe the underlying data
        1. Students will tease apart model fit and model utility.
1. Students will express probability models both mathematically and computationally.
    1. Students will employ probability models (computationally and analytically) to answer research questions.
1. Students will explain and implement different approaches for fitting probability models from data.
    1. Tuning
    1. Method of Moments
    1. Maximum likelihood
    1. Bayesian posterior
1. Students will visualize the uncertainty inherent in fitting probability models from data.
    1. Students will explore how to communicate uncertainty when constructing models and answering research questions.
    1. Students will propagate uncertainty in simulations
1. Students will explore the trade-offs of model complexity and generalizability.

</details>
<details>
<summary>4. </summary>
</details>
